\documentclass{article}
\usepackage[margin=2cm]{geometry}
\usepackage{float}
\usepackage{graphicx}


% Paragraph settings
\setlength{\parskip}{10pt plus 1pt minus 1pt
\setlength{\parindent}{0cm}}

\begin{document}
\title{CS22510 - Assignment 2 \\ A Discussion on Language Choices}
\author{Samuel Jackson \\ \texttt{slj11@aber.ac.uk}}
\date{\today}
\maketitle

\section{Discussion of the Choice of Languages Used}
In assignment one of this module, we were tasked to create three programs to perform different task to do with rider events using three different languages. Each of these three programs had differing requirements and therefore will be better suited to the features offered by each languages.

I chose to use the Java programming language for the checkpoint manager application. There are several reasons why i felt this was the best choice of language for this particular application, but the biggest reason was that Java has an extensive GUI library available as standard. This is useful because there is absolutely no set-up required to begin creating GUI applications as the Swing library is already bundled with the JDK. There are also several graphical tools to help create Swing GUIs rather than having to tediously hand code it.

Another major reason why I chose to create the checkpoint manager in Java was because of its easy deployment (including the GUI) on multiple platforms. Because compiled Java code executes on a JVM, all library features are supported out of the box, so there are no worries about cross platform compatibility. This is useful as the checkpoint manager application is designed to be run in the field, potentially on many different types of operating systems.

The wide choice of data structures also played a key part in my decision to use Java for this application. As I had already written the event manager in C, this application was the most complex to write. The standard containers, such as hashtables and priority queues, available in Java cut down the development time and increased the efficiency of the application.

For the event creation program I chose to use C++. The event creation program was the smallest of the three applications and being less familiar with C++ than Java and C, I decided to make this application with C++ as I had less knowledge of the facilities available with C++.

However, the development of the event creation program benefited from the Object Orientation of C++ to structure data input from users. As the program allows users to have multiple events, entrants and course instantiated at the same time, the power of classes (instead of alternatives such as structures in C) was a key reason why I chose C++ for this application.

Another key point for selecting C++ was that while it's standard library might not be quite a feature rich as Java's, it still provides a wide range of containers and helper functions, making development of the application substantially easier than C. It is also easier to work with file streams in C++ over C, which was very important in my decision since a large part of the functionality for the event creation program deals with formatted file creation.

While C++ and Java share many common features, the real pivotal point in my decision of which one to use for each application was that GUI development in C++ requires more set-up and is does not come as a standard part of C++. I also do not have any previous experience with a C++ GUI framework and therefore decided on Java for the checkpoint manager.

Finally, I chose to use C for the event manager program. The biggest reason for doing so was that the majority of the application's functionality had already been written using C for a previous assignment and therefore it was easier to keep most of the previous work and "bolt on" the new functionality (such as file locking). My choice was also heavily influenced by the relative difficulty in development of C programs compared with the more modern features of the other two languages.

\section{Comparison of the Languages Used}
Java, C++ and C all have very different features and characteristics that define them. My experience working with the three languages in the previous assignment has given me a good insight strengths and weaknesses of each.

While I ended up writing the most code for the Java application, comparatively I would of had to write a lot more code if I had chosen to use C or C++ instead. The reason that the Java application is the biggest is that it contains the most functionality out of all the programs in the assignment. This is because with the other two languages (particularly C) there are less standard data structures and library functions available than with Java. A good example of this is that C++ does not have a "Date" class like Java does, but instead provides some basic functions to parse strings into dates. While the same end result is achieved, more work (an hence more code) is involved on behalf of the programmer to get the same solution.

I have found the clarity of Java code to be extremely clear. Because of the aforementioned in built functionality of Java, it means that the programmer spends most of his/her time developing application logic code, rather than spending their time building less immediately relevant components such as having to write a linked list from scratch. With Java if you require a particular data structure there is usually a reasonably appropriate implementation already available. A good example from my code is the use of a priority queue to store checkpoint times.

Much of this is in sharp contrast to C program development. The C program language is relatively simple when compared to a larger, more modern language such as Java. It even lacks some of the basic types which almost all other high level languages provide (such as a boolean primitive or a string data type). In sharp contrast with both Java and C++, C is a functional language, meaning that it is often more difficult to organise code than with an object orientated language. Unlike the other two languages, C incorporates virtually no support for polymorphism and only very rudimentary encapsulation (if you class local vs. global scoped variables as encapsulation). Therefore the amount of code written in comparison to the level of functionality delivered is larger by comparison. The clarity of the code produced is often obscure and complicated as there are many "nasty" bits of C code that need to be avoided in order for a program to be maintainable. C code does not seem as readable as there object orientated counterparts and therefore good structure, organisation and documentation of C code is paramount.

However, there is one very important feature of the C programming language that distinguished it as being useful in comparison to Java. The C language gives the programmer full access to manual memory management features, allowing them to allocate and free memory as they require it. While this feature was not a huge benefit to the development of my application, it has powerful ramifications for programming anything where efficiency is paramount. Another use point about the C language is that it is extremely "low level" for a high level language. This is useful because it provides the programmer with almost direct access to the underlying operating systems facilities. While this can cause portability issues, it gives the programmer a higher degree of flexibility in some circumstances, for example when writing file I/O operations.

In comparison, C++ stands out as being mid way between both Java and C. It incorporates features present in both of the other languages. Being object orientated means that C++ allows the developer to create complex data structures easily, as well as utilising the power of polymorphism, encapsulation and inheritance. It also includes provides the manual memory management available in C with some additional improvements such as auto pointers and the new operator. This means that C++ and C, unlike Java, both do not have "garbage collection" facilitates to delete allocated memory when not being used. It is the programmers responsibility to allocation and deallocate memory. While this can be very useful, it can also be a hindrance as the developer must be careful to prevent memory leaks.

In contrast with C, C++ code seemed to require less code to be written. There is better support for basic types (i.e. string and boolean are provided) and there is better safety when using some potential dangerous features (e.g. auto pointers and the new keyword). There is also a decent set of standard containers available (such as vectors). This came incredibly useful in the event creation program as it meant that I did not have to write and manage my own dynamic storage structure (as with C). However, generally I found C++ to still require more code to be written when compared with Java, mostly due to the fact that the library provided by C++ is not quite so fully featured as Java's. I also found the clarity of the code written in C++ to be slightly less than compared with Java. This was partly due to the fact that there are simply less features available in C++ and therefore more code needs to be written, but also because C++ uses operators over and over again (like C), rather than defining new keywords, to help ensure backwards compatibility. A good example of this would be the dual meaning of the ampersand in C++.

\section{Conclusions}

\end{document}