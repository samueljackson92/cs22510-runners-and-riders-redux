\documentclass{article}
\usepackage[margin=2cm]{geometry}
\usepackage{float}
\usepackage{graphicx}


% Paragraph settings
\setlength{\parskip}{10pt plus 1pt minus 1pt
\setlength{\parindent}{0cm}}

\begin{document}
\title{CS22510 - Assignment 2 \\ A Discussion on Language Choices}
\author{Samuel Jackson \\ \texttt{slj11@aber.ac.uk}}
\date{\today}
\maketitle

\section{Discussion of the Choice of Languages Used}
In assignment one of this module, we were tasked to create three programs to perform different task to do with rider events using three different languages. Each of these three programs had differing requirements and therefore will be better suited to the features offered by each languages.

I chose to use the Java programming language for the checkpoint manager application. There are several reasons why i felt this was the best choice of language for this particular application, but the biggest reason was that Java has an extensive GUI library available as standard. This is useful because there is absolutely no set-up required to begin creating GUI applications as the Swing library is already bundled with the JDK. There are also several graphical tools to help create Swing GUIs rather than having to tediously hand code it.

Another major reason why I chose to create the checkpoint manager in Java was because of its easy deployment (including the GUI) on multiple platforms. Because compiled Java code executes on a JVM, all library features are supported out of the box, so there are no worries about cross platform compatibility. This is useful as the checkpoint manager application is designed to be run in the field, potentially on many different types of operating systems.

\section{Comparison of the Languages Used}

\section{Conclusions}

\end{document}